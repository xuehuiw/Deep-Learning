% !TeX program = xelatex
\documentclass[a4paper,12pt,fontset=windows]{ctexrep}

% ==================== 宏包引用 ====================
\usepackage{geometry}           % 页面布局
\usepackage{setspace}           % 行距设置
\usepackage{graphicx}           % 图片
\usepackage{float}              % 浮动体位置
\usepackage{booktabs}           % 三线表
\usepackage{array}              % 表格增强
\usepackage{amsmath}            % 数学公式
\usepackage{caption}            % 图表标题设置
\usepackage{pdfpages}           % 插入PDF
\usepackage{enumitem}           % 列表设置
\usepackage{fancyhdr}           % 页眉页脚
\usepackage{titletoc}           % 目录格式
\usepackage[hidelinks]{hyperref} % 超链接与书签(必须加载,提供\phantomsection)

% ==================== 页面设置 ====================
% 页面边距 (常规排版,符合大赛要求)
\geometry{left=3.17cm,right=3.17cm,top=2.54cm,bottom=2.54cm}

% 3. 排版:大纲级别为“正文文本”,两端对齐,首行缩进2字符,段前/段后0行,1.5倍行距
\setlength{\parindent}{2em}      % 首行缩进2字符
\setlength{\parskip}{0pt}        % 段前段后0行
\renewcommand{\baselinestretch}{1.5} % 1.5倍行距

% ==================== 字体设置 ====================
% 仅依靠 fontset=windows 自动配置。
% 它对应:中文正文=宋体,粗体=黑体,斜体=楷体。
% 西文正文=Times New Roman (若系统有)。

% ==================== 章节标题格式设置 ====================
% 4. 章节层级:第X章→X.X节→X.X.X小节→X.X.X.X子节
\setcounter{secnumdepth}{4}      % 编号深度到 subsubsection (X.X.X.X)
\setcounter{tocdepth}{4}         % 目录深度到 subsubsection

% 章标题格式:居中,黑体(或其他醒目字体),"第X章"
\ctexset{
    chapter = {
        name = {第,章},
        number = \arabic{chapter}, % 使用阿拉伯数字 1, 2, 3
        format = \centering\bfseries\zihao{3}, % 三号字体
        aftername = \quad,
        beforeskip = 12pt,
        afterskip = 12pt,
    },
    section = {
        format = \bfseries\zihao{4}, % 四号
        aftername = \quad,
    },
    subsection = {
        format = \bfseries\zihao{-4}, % 小四
        aftername = \quad,
    },
    subsubsection = {
        format = \bfseries\zihao{-4}, % 小四
        aftername = \quad,
    }
}

% ==================== 图表公式设置 ====================
\numberwithin{figure}{chapter}
\numberwithin{table}{chapter}
\numberwithin{equation}{chapter}

% 图表标题格式:小四号(与正文一致)
\DeclareCaptionFont{xiaosi}{\zihao{-4}}
\captionsetup{
    font={xiaosi},
    labelsep=quad, 
    labelfont={bf}, 
    textfont={normalfont},
    justification=centering
}
% 表头在上方,图名在下方
\captionsetup[table]{position=top}
\captionsetup[figure]{position=bottom}

% ==================== 参考文献设置 ====================
% 重定义参考文献列表的标签格式为 [1]
\makeatletter
\renewcommand\@biblabel[1]{[#1]}
\makeatother

% ==================== 文档开始 ====================
\begin{document}

% ==================== 封面 ====================
\begin{titlepage}
    \centering
    \vspace*{2cm}
    \Huge\bfseries 2026年中国大学生机械工程创新创意大赛\\
    \vspace{0.5cm}
    \Huge\bfseries 铸造工艺设计赛\\
    \vspace{3cm}
    \Huge\bfseries 参\quad 赛\quad 作\quad 品\\
    
    \vspace{4cm}
    \Large\bfseries
    \begin{tabular}{rc}
        铸件名称:& \underline{\makebox[8cm]{XXXXXX}} \\[1cm]
        自编代码:& \underline{\makebox[8cm]{XXXXXX}} \\
    \end{tabular}
    
    \vspace*{\fill}
    \centering
    \Large 2026年5月
\end{titlepage}

% ==================== 摘要 ====================
\pagenumbering{Roman} % 摘要页码罗马数字
\chapter*{摘\quad 要}
\addcontentsline{toc}{chapter}{摘要}

\begin{spacing}{1.5}
    % 正文:400-600字,需涵盖零件名称、材质、结构分析、生产技术要求、工艺方案等
    本文针对XXXX零件进行了铸造工艺设计。该零件材质为XXXX,结构呈XXXX特征。通过对零件结构的工艺性分析,确定了生产技术要求,包括XXXX。本文制定了XXXX工艺方案,详细设计了浇注系统、冒口及冷铁等工艺参数。\underline{\hspace{3cm}}(此处为摘要内容,需补充至400-600字)\underline{\hspace{3cm}}通过模拟仿真优化,消除了潜在的缩孔与裂纹缺陷,最终确定的工艺方案切实可行。
\end{spacing}

\vspace{1em}
\noindent\textbf{关键词:}关键词1,关键词2,关键词3

\newpage

% 目录
\tableofcontents
\newpage

% 正文开始
\pagenumbering{arabic} % 正文页码阿拉伯数字

% 第1章
\chapter{绪论}
此处为第1章正文内容。

% 第2章
\chapter{工艺方案设计}

\section{图表示例}

\subsection{图片格式要求}
图需主辅线分明,字符/数据准确。随文放置(先见文后见图)。

\begin{figure}[H]
    \centering
    % 使用 \framebox 模拟图片,实际使用:\includegraphics[width=0.8\textwidth]{filename.jpg}
    \framebox[0.8\textwidth]{\rule{0pt}{5cm}在此处插入图片}
    \caption{铸件三维结构示意图}
    \label{fig:example}
    \vspace{0.2em}
    \footnotesize 注:本图基于SolidWorks 2024建模。
\end{figure}

\subsection{表格格式要求}
表头+表格内容。可加“注”、“数据来源”。

\begin{table}[H]
    \centering
    \caption{工艺参数表}
    \label{tab:example}
    \begin{tabular}{ccccc}
        \toprule
        参数名称 & 数值   & 单位        & 备注   & 数据来源 \\
        \midrule
        浇注温度 & 1450 & $^\circ$C & 关键参数 & 查表   \\
        充型时间 & 15   & s         &      & 计算   \\
        \bottomrule
    \end{tabular}
    \vspace{0.2em}
    \footnotesize 数据来源:GB/T XXXXX。
\end{table}

% 续表格式示例
若表格跨页,请使用如下格式手动创建续表:

\begin{table}[H]
    \centering
    \caption*{续表\thechapter.\arabic{table} 工艺参数表(续)}
    \begin{tabular}{ccccc}
        \toprule
        参数名称 & 数值             & 单位 & 备注 & 数据来源 \\
        \midrule
        砂箱尺寸 & 500$\times$500 & mm &    & 手册   \\
        \bottomrule
    \end{tabular}
\end{table}

\section{公式格式}
公式需右上角标注编号(如(2.1))。下方用“式中”注明各符号含义。

\begin{equation} \label{eq:example}
    t = K \sqrt{G}
\end{equation}

\noindent 式中,$t$——凝固时间(min);\\
\hspace*{2.5em}$K$——凝固系数(mm/min$^{0.5}$);\\
\hspace*{2.5em}$G$——折算厚度(mm)。

\section{层级测试}
\subsection{二级标题}
\subsubsection{三级标题}
\paragraph{四级标题} % latex 中 subsubsection 之后是 paragraph

\chapter{总结}
总结内容。

% 参考文献
\clearpage
\phantomsection
\addcontentsline{toc}{chapter}{参考文献}

\chapter*{参考文献}
\vspace{-1em}

\begin{thebibliography}{99}
    \setlength{\itemsep}{0pt}

    % 1. 专著
    \bibitem{ref1} 张三,李四.铸造工艺学[M].北京:机械工业出版社,2020.10-25

    % 2. 译著
    \bibitem{ref2} 约翰·史密斯.现代铸造技术[M].王五 译.上海:上海科学技术出版社,2019.50-60

    % 3. 期刊文章
    \bibitem{ref3} 赵六. 铝合金铸造缺陷分析[J]. 铸造,2023,72(3):35-39

    % 4. 标准
    \bibitem{ref4} GB/T 11351-2017,铸件重量公差[S]

    % 5. 专利
    \bibitem{ref5} 某某公司. 一种消失模铸造方法[P]. 中国:CN101234567A,2021-05-20

\end{thebibliography}

% 附图
\clearpage
\phantomsection
\addcontentsline{toc}{chapter}{附图}

\chapter*{附\quad 图}
\noindent 附图1:\underline{\hspace{5cm}}(附图名称)
\begin{figure}[H]
    \centering
    \framebox[0.9\textwidth]{\rule{0pt}{10cm}PDF附图占位}
    % \includegraphics[width=0.9\textwidth]{futu1.pdf}
    \caption{附图名称}
\end{figure}

\end{document}
